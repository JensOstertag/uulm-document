\documentclass[a4paper]{uulm-document}

%%%%%%%%%%%%%%%%%%%%%%%%%%
%   German Date Format   %
%%%%%%%%%%%%%%%%%%%%%%%%%%
\usepackage[ngerman]{babel}

%%%%%%%%%%%%%%%%%%%%%%%%%%%%
%   Document Information   %
%%%%%%%%%%%%%%%%%%%%%%%%%%%%
\documentTitle{Titel des Dokuments}
\documentSubtitle{Subtitel des Dokuments}
\documentAuthor{Autor}
\documentDate{\today}

%%%%%%%%%%%%%%%%%%%%%%%%%%%%%%%%%%%%%%%%%%%%%%%%%%%%%%%%%%%%%%%%%
%                            Faculty                            %
%   Available Faculties: "med", "nat", "infIngPsy" and "math"   %
%%%%%%%%%%%%%%%%%%%%%%%%%%%%%%%%%%%%%%%%%%%%%%%%%%%%%%%%%%%%%%%%%
\faculty{infIngPsy}

%%%%%%%%%%%%%%%%%%%%%%%%%%%%%%%%%%%%%%%%%%%%
%             Title Page Style             %
%   Available Styles: "default", "clear"   %
%%%%%%%%%%%%%%%%%%%%%%%%%%%%%%%%%%%%%%%%%%%%
\titlepagestyle{default}

\begin{document}
\maketitle
\tableofcontents

\section{Abschnitte}
Das ist eine \textit{section}. Sie wird im Inhaltsverzeichnis aufgeführt. In einer pdf-Datei ist sie so verlinkt, dass man mit einem Klick auf die Zeile direkt zum entsprechenden Abschnitt gelangt.

Auch \textit{subsection}s werden im Inhaltsverzeichnis aufgelistet und entsprechend verlinkt.

\section{Formeln}
Mit \LaTeX\, können schöne Formeln einfach dargestellt werden:
\begin{equation}
p_{ph} = \frac{h}{\lambda} = \frac{h*f}{c} = m * c
\end{equation}
Mit dieser Formel kann man zum Beispiel den Impuls eines Photons berechnen!

\section{Graphen}
\begin{center}
\begin{tikzpicture}
\begin{axis}[
	minor tick num=1,
	axis lines=middle,
	xmin=-6.5,
	xmax=6.5
]
\addplot[domain=-2*pi:2*pi, color=facultyColor, samples=30, smooth]{sin(deg(x))};
\end{axis}
\end{tikzpicture}
\end{center}
Auch Graphen lassen sich mit \LaTeX\, schön abbilden. Dafür wird das Paket \textit{pgfplots} genutzt, wie oben mit der Sinus-Funktion gezeigt. Wird die Funktion mit der Farbe \textit{facultyColor} gezeichnet, erscheint sie automatisch mit der Farbe der ausgewählten Fakultät.

\textit{pgfplots} erlaubt es sogar, dreidimensionale Graphen zu zeichnen, hier am Beispiel $sin(x)-cos(y)$ veranschaulicht:
\begin{center}
\begin{tikzpicture}
\begin{axis}[
	minor tick num=1,
	view={20}{60}
]
\addplot3[surf, samples=45]{sin(deg(x))-cos(deg(y))};
\end{axis}
\end{tikzpicture}
\end{center}

\section{Programmcodes}
Mit dieser \LaTeX -DocumentClass können ebenfalls Programmcodes mit Code Highlighting dargestellt werden:
\begin{lstlisting}[language=Java]
public class Main {
    public static void main(String args[]) {
        int[] array = new int[10];
        for(int i = 0; i < array.length; i++)
            array[i] = 1 + (int)(Math.random() * 10);
        
        int[] sorted = sort(array);
        
        String initialOutput = "Starting Array: ";
        String sortedOutput = "Sorted Array: ";
      
        for(int i = 0; i < sorted.length; i++) {
            initialOutput += array[i] + " ";
            sortedOutput += sorted[i] + " ";
        }
      
        System.out.println(initialOutput + "\n" + sortedOutput);
    }
    
    public static int[] sort(int[] array) {
        int[] sorted = new int[array.length];
        
        for(int i = 0; i < sorted.length; i++)
            sorted[i] = array[i];
            
        for(int i = 0; i < sorted.length; i++) {
            for(int j = i; j < sorted.length; j++) {
                if(sorted[i] > sorted[j]) {
                    int cache = sorted[j];
                    sorted[j] = sorted[i];
                    sorted[i] = cache;
                }
            }
        }
        
        return sorted;
    } 
}
\end{lstlisting}
\end{document}