\documentclass[a4paper]{../uulm-document}

%%%%%%%%%%%%%%%%%%%%%%%%%%%%
%   Document Information   %
%%%%%%%%%%%%%%%%%%%%%%%%%%%%
\documentTitle{uulm-document}
\documentSubtitle{Documentation}
\documentAuthor{Jens Ostertag}
\documentDate{\today}

%%%%%%%%%%%%%%%%%%%%%%%%%%%%%%%%%%%%%%%%%%%%%%%%%%%%%%%%%%%%%%%%%
%                            Faculty                            %
%   Available Faculties: "med", "nat", "infIngPsy" and "math"   %
%%%%%%%%%%%%%%%%%%%%%%%%%%%%%%%%%%%%%%%%%%%%%%%%%%%%%%%%%%%%%%%%%
\faculty{infIngPsy}

%%%%%%%%%%%%%%%%%%%%%%%%%%%%%%%%%%%%%%%%%%%%
%             Title Page Style             %
%   Available Styles: "default", "clear"   %
%%%%%%%%%%%%%%%%%%%%%%%%%%%%%%%%%%%%%%%%%%%%
\titlepagestyle{clear}

%%%%%%%%%%%%%%%%%%%%%%%%%%%%%%%%%%%%%%%%%
%   Print Mode (uncomment applicable)   %
%%%%%%%%%%%%%%%%%%%%%%%%%%%%%%%%%%%%%%%%%
%\printmodetrue
%\printmodefalse

\lstset
{
    language=[LaTeX]TeX,
    keywordstyle=\color{blue},
    identifierstyle=\color{magenta},
}

\begin{document}
\maketitle
\tableofcontents

\section{Introduction}
This is the Documentation to the \LaTeX -Documentclass \textit{uulm-document}, an inofficial \LaTeX -Template for Documents at Ulm University. It contains Details about used Packages, Commands and Colors.

For Updates to this Class, have a Look at the \href{https://github.com/JensOstertag/uulm-document}{GitHub Page}.

\subsection{Used Packages}
This Documentclass needs to import a few Packages that are required to compile the Document:
\begin{itemize}
\item \textbf{ifthen} - used for Command Handling within the Documentclass
\item \textbf{geometry} - used to set the Margin of the Document to 1in
\item \textbf{inputenc} - used for UTF-8 Encoding
\item \textbf{xcolor} - used to define Colors
\item \textbf{hyperref} - used to create Hyperlinks within the Table of Contents (you can use it for your own Hyperlinks too)
\item \textbf{tocloft} - used to draw dotted Lines within the Table of Contents
\item \textbf{graphicx} - used to place the UUlm Logo (you can use it for your own Images too)
\item \textbf{scrlayer-scrpage} - used to place the Headers
\item \textbf{tikz} - used to draw the Title Page and the Headers (you can use it for your own Drawings, too)

There are also these TikZ-Packages included:
\begin{itemize}
\item calc
\item positioning
\item fit
\end{itemize}
They are all used to style the Title Page and the Headers.
\item \textbf{mathtools} - used to create the \textit{qed} Command
\end{itemize}
Additionally, the following Packages are included and set up to allow a better overview in your Main Document:
\begin{itemize}
\item \textbf{pgfplots} - used to draw Graphs in tikzpictures
\item \textbf{amsmath} - used for better Equations
\item \textbf{amssymb} - provide an extended Collection of mathematical Symbols
\item \textbf{listings} - used to create Source Codes
\item \textbf{csquotes} - used for enquoting Text
\item \textbf{imakeidx} - used to create an Index at the end of the Document
\end{itemize}

\subsection{Commands}
There are some custom Commands included in this Documentclass:
\begin{itemize}
\item \textbf{documentTitle} - sets the Title of the Document shown on the Title Page
\item \textbf{documentSubtitle} - sets the Subtitle of the Document shown on the Title Page
\item \textbf{documentAuthor} - sets the Author of the Document shown on the Title Page
\item \textbf{documentDate} - sets the Date of the Document shown on the Title Page
\item \textbf{faculty} - sets the Faculty and the corresponding Faculty Color
\item \textbf{setFacultyColor} - changes the Faculty Color to any given Color (Hex Format)
\item \textbf{setAccentColor} - changes the Accent Color to any given Color (Hex Format)
\item \textbf{titlepagestyle} - sets the Title Page Type
\item \textbf{printmodetrue} - enables Print Mode
\item \textbf{printmodefalse} - disables Print Mode
\item \textbf{qed} - prints a qed Square
\end{itemize}

\subsection{Colors}
There are a few Colors included to this Documentclass, some of them are constant, and some of them are changing, depending on your Document Settings. 
The static Colors are:
\begin{itemize}
\item \textbf{medColor} - {\color{medColor}\#26547C}
\item \textbf{natColor} - {\color{natColor}\#DF6D07}
\item \textbf{infIngPsyColor} - {\color{infIngPsyColor}\#A32638}
\item \textbf{mathColor} -{\color{mathColor}\#56AA1C}
\item \textbf{defaultColor} - {\color{defaultColor}\#7D9AAA}
\item \textbf{codeBackgroundColor} - {\color{black}\#FFFFFF}
\item \textbf{codeCommentColor} - {\color{codeCommentColor}\#7F7F7F}
\item \textbf{codeKeywordColor} - {\color{codeKeywordColor}\#00B200}
\item \textbf{codeNumberColor} - {\color{codeNumberColor}\#000000}
\item \textbf{codeStringColor} - {\color{codeStringColor}\#FF4C00}
\end{itemize}
The dynamically changing Colors are:
\begin{itemize}
\item \textbf{facultyColor} - Depending on your selected Faculty and Overrides on this Color (setFacultyColor), Default {\color{defaultColor}\#7D9AAA}
\item \textbf{accentColor} - Depending on Overrides on this Color (setAccentColor), Default {\color{black}\#CBC6BA}
\end{itemize}

\subsection{Fonts}
The default Font Family was replaced by \textit{sfdefault}.

\newpage

\section{Usage}
The following Instruction explains how to use the Documentclass and it's Features.
\subsection{Getting Started}
At first, you may want to \textbf{create} a \textbf{Project Folder} on your Device, where you keep all Files regarding your Document.

Then, please \textbf{download} the \textbf{current Version} from \href{https://github.com/JensOstertag/uulm-document}{GitHub} and unpack the Files. Next, you have to \textbf{move} the \textbf{File} \textit{uulm-document.cls} and the \textbf{Directory} \textit{img} \textbf{to} your \textbf{Project Folder}.

{\setlength\parindent{24pt} \color{gray}\textit{uulm-document.cls} is the Documentclass File and the Directory \textit{img} contains all Images.

It is suggested to put all of your Images there, too.}

As the last Step you can \textbf{create} the \textit{.tex} \textbf{File} for your Document.

\subsection{Document Setup}
Next, let's have a look at the Code you have to write. There is already an example Document available in the \href{https://github.com/JensOstertag/uulm-document}{GitHub Repository}, but this Instruction will describe the used Commands more detailed.

You have to start by defining a Documentclass. This should be, of course, \textit{uulm-document}, that can be selected with
\begin{lstlisting}
\documentclass{uulm-document}
\end{lstlisting}
You could also add Class Options, e.g. to change the Size of the Page:
\begin{lstlisting}
\documentclass[a4paper]{uulm-document}
\end{lstlisting}

The next Task will be to set up your Document Metadata: Tile, Subtitle, Author Name and the Date. You can use the following Commands to do that:
\begin{lstlisting}
\documentTitle{Document Title}
\documentSubtitle{Document Subtitle}
\documentAuthor{Author Name}
\documentDate{Date}
\end{lstlisting}
To use the current Date as the Date in the Document, you can type \lstinline!\date! instead of \lstinline!Date!.

Then, you can set up the Document to your own Likings:
\begin{itemize}
\item \textbf{Select} your \textbf{Faculty}

Available Faculties are \textit{med} (Medicine), \textit{nat} (Natural Sciences), \textit{infIngPsy} (Information Technology, Engineering, Psychology) and \textit{math} (Mathematics). 

If you don't set your Faculty, or don't set it correctly, the Document will be created with a light Blue Color.

You can set your Faculty by using
\begin{lstlisting}
\faculty{med | nat | infIngPsy | math}
\end{lstlisting}
(Remove all except relevant).

\item \textbf{Select} the \textbf{Title Page Style}

Available Title Page Styles are \textit{default} and \textit{clear}.

A \textit{default} Title Page Style will start your Document Contents right below the Document Title. Using \textit{clear} Title Page Style will clear the Title Page and your Document Contents will start on the next Page.

You can set a Title Page Style by using
\begin{lstlisting}
\titlepagestyle{default | clear}
\end{lstlisting}
(Remove all except relevant).

\item \textbf{Select} the \textbf{Print Mode}

Enabling the Print Mode in your Document will remove the colored Headers and print the Page Number in the Center at the Bottom of each Page. This could be used if you want to print your Document to reduce Color Usage.

By Default, Print Mode is turned off.

You can enable the Print Mode by using
\begin{lstlisting}
\printmodetrue
\end{lstlisting}
\end{itemize}

\subsection{Adding Contents to the Document}
Now, you can add Contents to your Document inside 
\begin{lstlisting}
\begin{document}
	% Your Contents Here
\end{document}
\end{lstlisting}
You may want to start with creating the Title Page. To do that, use the Command
\begin{lstlisting}
\maketitle
\end{lstlisting}
There won't be a Page Number on the Title Page.

Next, you can create a Table of Contents with
\begin{lstlisting}
\tableofcontents
\end{lstlisting}
that will list all your \textit{Section}s, \textit{Subsection}s and \textit{Subsubsection}s together with the corresponding Page Number.

All the other Contents are the same as in normal \LaTeX\;Code.

\end{document}